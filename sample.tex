% Example from below
% https://www.overleaf.com/learn/latex/Learn_LaTeX_in_30_minutes
\documentclass[12pt, A4]{article} %  defines the overall class (type) of document.
%if we use book class instead of article, we can use \chapter{Let's begin} before  \section{•}
\usepackage{graphicx} %LaTeX package to import graphics
%\usepackage[bottom=0.5cm, right=1.5cm, left=1.5cm, top=1.5cm]{geometry} % to change the margin of Page
\graphicspath{{images/}} %configuring the graphicx package


% ///////Below is components for title page ///////////
\title{A Beginner's Guide to DGUS V7.641}
\author{Sannan Ahmed}
\date{\today} %\date{August 2022}
%/////////////////////////////


% any thing above \begin is called preamble

\begin{document}

%if you press two "enter" key then new paragraph will created

% ///////this is to make a title page///////////
\begin{titlepage}
\maketitle
\thispagestyle{empty} %this will clear the number from the title page
\end{titlepage}
%////////////////////////////////

%to add a table of contents (table of contents need to be updated (cut then Quick Build & paste then Quick Build) if any content changes.

\tableofcontents
%to add new line
\newpage
\section{Flash Memory}
The LCD have total of 16MBytes(Mega Bytes) Flash memory for all content including fonts,images,icons,animation etc. This memory is divided into 64 id or subspaces, 0 to 63, each of 256KB size.

The files in the DWIN\_SET folder are stored here for execution. each file in the folder occupies the subspace correspoding to its name, e.g the '1.bin' files goes to the 1 id/subspace and '32.icl' to 32 and so on. If the file is larger then 256KB it will blead to the next regin. For example if '32.icl' is 1 MB, then it will occupie the 32,33,34 and 35 regin.
\begin{figure}[!htb] %[!htb] is used to place image where it is in editor
	\centering
	\includegraphics[width=14cm]{flashMem} 
	\caption{Dwin LCD memory spaces, and thire usage}
\end{figure}
\section{Gray word font}
\subsection{Creating Font}
Open DGUS Software, go to welcome '1' and select the grey font generator '2'.
\begin{figure}[!htb] %[!htb] is used to place image where it is in editor
	\centering
	\includegraphics[width=14cm]{grayFont1} 
	\caption{Dgus software, open the gray font creator.}
\end{figure}
Select Font '1' to use and then enable the sizes whcih you want to use '2', the horizontal size is same as the index (size of font) and the vertical is 2 times of horizontal size. You can note which sizes you have with these valves to use in DGUS software. '3' save the font as '1.bin' this will be used as 'Font ID' in software.
\begin{figure}[!htb] %[!htb] is used to place image where it is in editor
	\centering
	\includegraphics[width=14cm]{grayFont2} 
	\caption{Creation process of grayfont}
\end{figure}

It is created using software Gray word library generator from DWIN. It has greater experience than 0 word font. For gray word font we have to use 8bit encapsulation.

% The \includegraphcs command is provided (implemented) by the graphicx package
\begin{figure}[!htb] %[!htb] is used to place image where it is in editor
	\centering
	\includegraphics[width=14cm]{grayFont} 
	\caption{This picture shows necessary setting.}
\end{figure}

\begin{figure}[!htb] %[!htb] is used to place image where it is in editor
	\centering
	\includegraphics[width=14cm]{encoding} 
	\caption{This picture encoding settings.}
\end{figure}


%this is how to go on new page
\newpage


\section{data send serially}

\emph{Note: First of all enable Touch variable automatic upload 1=On from system configuration. This is needed to send data from Syncrodata return and return key code functions of DGUS software}

\begin{figure}[!htb] %[!htb] is used to place image where it is in editor
	\centering
	\includegraphics[width=10cm]{autoupload} 
	\caption{auto Upload commmand}
\end{figure}

following image shows the command through which data can be send/receive from MCU. 
\\ Following command will change the page. (it will make use of system variable)


\begin{figure}[!htb] %[!htb] is used to place image where it is in editor
	\centering
	\includegraphics[width=10cm]{receive} 
	\caption{send/receive from MCU command (Data received is : 5523)}
\end{figure}

\newpage


\section{page change from synchrodata return}

Page can be changed from synchodata return. see the below image.

\begin{figure}[!htb] %[!htb] is used to place image where it is in editor
	\centering
	\includegraphics[width=14cm]{pageChageCommand} 
	\caption{pageChageCommand}
\end{figure}

\newpage

\section{Save variable in Nor Flash}

Address of nor flash, it must be an even number, the range is 0x000000 - 0x027FFE, and then one address corresponds to 2 bytes, that is the total capacity is 320KB.\\

\emph{for NOR flash database site \\
Each ID corresponds to 2KWords memory with ID range of 0 - 79.
The database is located in on-chip NOR FLASH of 160KWords (320 KB). It
can be used to save user data or DWIN OS program library files.}


\begin{figure}[!htb] %[!htb] is used to place image where it is in editor
	\centering
	\includegraphics[width=14cm]{NorFlash} 
	\caption{NorFlash}
\end{figure}

\newpage

\end{document}